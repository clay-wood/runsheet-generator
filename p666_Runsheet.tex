\documentclass[letterpaper, 10pt]{article}
\usepackage[table,xcdraw]{xcolor}
\usepackage{textcomp}
\usepackage{gensymb}
\usepackage{amsmath}  % improve math presentation
\usepackage[left=0.75in,top=0.75in,bottom=0.2in,letterpaper]{geometry} % decreases margins
\usepackage{setspace}
\usepackage{enumitem}


%----------------------------------------------


\begin{document}

\begin{center}
    {\Large \textbf{Biax Experiment}}\\
    {\small For current calibrations -- \texttt{gpfs/group/cjm38/default/Calibrations/}}\\
    {\footnotesize \textit{Revised: 30 Nov. 2021}}
\end{center}



\begin{table}[!ht]
	\renewcommand{\arraystretch}{1.1}
	\begin{tabular}{p{10cm} p{10cm} }
	    \textbf{Exp. Name: }p666 & \textbf{Date/Time: }clay\\
	    \textbf{Operator(s): }04/12/2021 & Hydraulics start: 0 \\
	    Temperature ($\degree$C): 0 & Hydraulics end: 0 \\
	    Relative Humidity ($\%$): 0 & Data Logger/Control File: 16-chan 0\\
	\end{tabular}
\end{table} 
\vspace{-0.5cm} 

\begin{table}[!ht]
	\renewcommand{\arraystretch}{1.1}
	\begin{tabular}{p{20cm}}Sample Block Used and Thickness with \textbf{no} Sample: Steel 5x5 cm \\
	\end{tabular}
    \end{table} \vspace{-0.5cm} 

\begin{table}[!ht]
        \small
        \renewcommand{\arraystretch}{1.2}
        \begin{tabular}{ |p{7cm}| } \hline 
Material: WG \\Particle Size, Distribution: 0 \\Benchtop Sample Thickness (mm): 0 \\\end{tabular} \end{table} \vspace{-0.5cm} 

\begin{table}[ht!]
        \renewcommand{\arraystretch}{1.5}
        \begin{tabular}{ |p{2.75cm}|p{4cm}|p{3.5cm}|p{2.5cm}| p{3cm}| }
            \multicolumn{3}{l}{\textbf{\textit{Load Cells:}}} & \multicolumn{2}{l}{Contact Area:  $ m^2 $}\\ \hline
            \textbf{Load cell name} & \textbf{Calibrations (mV/kN)} & \textbf{Target stress (MPa)} & \textbf{Init. Voltage} & \textbf{Volt. @ load}\\
            \hline
            


    \end{tabular}
    \end{table} \vspace{-0.5cm} 





\end{document}