\documentclass[letterpaper,10pt]{article}
\usepackage{blindtext}
\usepackage{authblk}
\usepackage{tabularx} % extra features for tabular environment
\usepackage{textcomp}
\usepackage{gensymb}
\usepackage{amsmath}  % improve math presentation
\usepackage{xcolor}
\usepackage{rotating}
\usepackage[margin=0.75in,letterpaper]{geometry} % decreases margins
\usepackage{setspace}
\usepackage{multirow}
\usepackage{enumitem}
\setlength{\arrayrulewidth}{0.25mm}
\renewcommand{\arraystretch}{1.5}
%++++++++++++++++++++++++++++++++++++++++

\begin{document}

\begin{center}
	{\Large \textbf{Biax Experiment}}
\end{center}	
\bigskip

\renewcommand{\arraystretch}{1}
\begin{tabular}{ p{11cm} p{10cm} }
    \textbf{Exp. Name: }p5458WGcutPerm5-20 & \textbf{Date/Time: }01 Nov. 2020 \\
    \textbf{Operator(s): }Wood & \textbf{Hydraulics start: }3613.2 \\
    & \textbf{Hydraulics end: }
\end{tabular}
    \bigskip 

\textit{Sample Block Thickness w/ no gouge: } 
\bigskip

\textit{Layer Thickness (total on bench):} mm @sample

\textit{Under Load: mm}

\textit{Material (Qtz, Granite, ?): }WG, Saw-cut \& 80-grit.

\textit{Particle Size, Size Distribution :}
\bigskip 

\renewcommand{\arraystretch}{1}
\begin{tabular}{ p{11cm} p{10cm} }
    \textbf{\textit{Load Cells:}} & Contact Area: 0.0022292545 $ m^2 $ \\
\end{tabular}


\renewcommand{\arraystretch}{1.5}
\begin{tabular}{ |p{2.75cm}|p{4cm}|p{3.5cm}|p{2.5cm}| p{3cm}| }
    \hline
    \textbf{Load cell name} & \textbf{Calibrations (mV/kN)} & \textbf{Target stress (MPa)} & \textbf{Init. Voltage} & \textbf{Volt. @ load}\\
    \hline
    44 mm Horiz. & HG: 119.303 & 5, 20 & 0.79220 & 2.122 , 6.1113\\ 
    \hline
\end{tabular}
\bigskip 

        \renewcommand{\arraystretch}{1}
    \begin{tabular}{ p{11cm} p{10cm} }
        \textbf{\textit{Vessel Pressure:}} & Pore Fluid: DI H2O \
    \end{tabular}

    \renewcommand{\arraystretch}{1.5}
    \begin{tabular}{ p{1cm}|p{4cm}|p{4.75cm}|p{2.5cm}| p{3.5cm}| }
        \cline{2-5}
        & \textbf{Calibrations $ (V/MPa) $} & \textbf{Pressures (MPa)} & \textbf{Init. Voltage} & \textbf{Volt. @ load}\\
        \cline{1-5}\multicolumn{1}{ |c| } {\textbf{Pc}} & Gain: 0.1456 & 3.145 & -0.089760 & 0.36815\\ 
            \hline\multicolumn{1}{ |c| } {\textbf{PpA}} & HG: 1.517 & 2.5, 2.5, 2.5, 2.5 & -0.088558 & 3.70394, 3.70394, 3.70394, 3.70394\\ 
            \hline\multicolumn{1}{ |c| } {\textbf{PpB}} & HG: 1.483 & 0.5, 1.0, 1.5, 2.0 & -0.046551 & 0.69495, 1.43645, 2.17795, 2.91945\\ 
            \hline
    \end{tabular}
        \medskip 

\renewcommand{\arraystretch}{1}
\begin{tabular}{ p{11cm} p{10cm} }
	\textbf{Data Logger Used: }16 channel &\textbf{Control File: }No  \\
\end{tabular}
\medskip 

\begin{tabular}{ p{11cm} p{10cm} }
	\textbf{Horiz. DCDT:} \textit{Short Rod} & \textbf{Vert. DCDT: }TT2 \\
	HG: 0.64 mm/V &
\end{tabular}
\medskip 

\textit{Purpose/Description: }
Permeability test of saw-cut sample roughened with 80-grit. \\ Sample to be sent to Andy Rathbun at Chevron for profilometry before DAET/PP osc. experiment. 
\\ 
 
\textit{Acoustics Blocks used: }
SDS L-block v2

\newpage 
 \textbf{Experiment Notes}
 \medskip
 \begin{itemize}[label=\#]
\end{itemize} 

 \end{document}